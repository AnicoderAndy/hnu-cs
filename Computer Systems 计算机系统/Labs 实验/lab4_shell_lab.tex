\documentclass[a4paper, 11pt]{ctexart}
\usepackage[top=2.54cm, bottom=2.54cm, left=3.18cm, right=3.18cm]{geometry}
\usepackage{titlesec}
\usepackage{hyperref}
\usepackage{setspace}
\usepackage{fancyhdr}
\usepackage{lastpage}
\usepackage{minted}
\usepackage{amsmath}
\usepackage{lmodern}

% stretch line spacing
\setstretch{1.5}
\AtBeginEnvironment{minted}{\singlespacing}
\AtBeginEnvironment{equation}{\singlespacing}

% titlesec settings
\let\oldsection\section
\renewcommand{\section}{\clearpage\oldsection}

% ctex settings
\ctexset{
    section = {
        format = \raggedright\zihao{3}\heiti
    }
}

% fancyhdr settings
\pagestyle{fancy}
\fancyhf{}
\setlength{\headheight}{1.5cm}
\fancyhead[C]{ShellLab实验报告}
\renewcommand{\headrulewidth}{0.5pt}
\fancyfoot[C]{\textbf{\thepage\quad/\quad\pageref{LastPage}}}

% minted settings
\setminted{
    linenos,
    frame = single
}

%%%%%%%%%%%%%%%%%%%%%%%%%%%%%%%%%%%%%%%%%%%%%%%%%%
% Begin Document
%%%%%%%%%%%%%%%%%%%%%%%%%%%%%%%%%%%%%%%%%%%%%%%%%%
\begin{document}
\sloppy
%%%%% Cover Page %%%%%
\begin{titlepage}
    \begin{center}
        \vspace*{5cm}
        \yahei\zihao{1}《计算机系统》\\ \vspace{50pt}ShellLab实验报告
        \vfill
        \zihao{3}
        \begin{tabular}{ll}
            班级: & GitHub\\
            学号: & 78268851\\
            姓名: & AnicoderAndy\\
        \end{tabular}
    \end{center}
\end{titlepage}
\setcounter{page}{2}

%%%%% Table of Contents %%%%%
\renewcommand{\contentsname}{\centering 目录}
\tableofcontents

\section{实验项目}
\setcounter{subsection}{0}
\subsection{项目名称}
Shell Lab——实现简易Shell程序

\subsection{实验目的}
\begin{itemize}
    \item 熟悉Linux Shell的基本概念。
    \item 熟悉Linux的异常控制流。
    \item 熟悉Linux的进程控制、进程间通信的基本方法。
\end{itemize}

\subsection{实验资源}
\begin{itemize}
    \item 实验环境:Linux Ubuntu 22.04 LTS
    \item 实验工具:Visual Studio Code、GNU Make 4.3、GCC 11.4.0
    \item 其他资源:实验手册、16个测试文件以及4个测试代码、测试脚本、\texttt{tsh.c}。
\end{itemize}

\section{实验任务}

本实验要求实现一个建简易的Shell程序,要求支持执行程序、执行\texttt{quit}、\texttt{jobs}、\texttt{fg}、\texttt{bg}四个内置命令。

本实验提供的代码实现了一些如解析命令行\texttt{parseline}、\texttt{sigquit}等简单的函数,需要实现:

\begin{minted}{c}
void eval(char* cmdline);
int builtin_cmd(char** argv);
void do_bgfg(char** argv);
void waitfg(pid_t pid);

void sigchld_handler(int sig);
void sigtstp_handler(int sig);
void sigint_handler(int sig);
\end{minted}

\subsection{便利工具函数}

为了后续代码方便,实现\texttt{verbose\_printf}方便进行开启\texttt{-v}情况下的输出:

\begin{minted}{c}
inline int verbose_printf(const char* fmt, ...) {
    if (verbose) {
        va_list args;
        va_start(args, fmt);
        int ret = vprintf(fmt, args);
        va_end(args);
        return ret;
    }
    return -1;
}
\end{minted}

由于\texttt{SIGCHLD}信号的处理函数和\texttt{waitfg}函数中都需要进行工作回收,因此实现了一个二者都可调用的工作回收函数:

\begin{minted}{c}
void job_reaping(pid_t pid, int status) {
    if (WIFEXITED(status)) {
        deletejob(jobs, pid);
        verbose_printf("job_reaping: Job [%d] (%d) deleted\n",
                       pid2jid(pid), pid);
    } else if (WIFSIGNALED(status)) {
        int jid = pid2jid(pid);
        int sig = WTERMSIG(status);
        deletejob(jobs, pid);
        verbose_printf("job_reaping: Job [%d] (%d) deleted\n", jid,
                       pid);
        printf("Job [%d] (%d) terminated by signal %d\n", jid, pid,
               sig);
    } else if (WIFSTOPPED(status)) {
        int jid = pid2jid(pid);
        int sig = WSTOPSIG(status);
        struct job_t* job = getjobjid(jobs, jid);
        job->state = ST;
        printf("Job [%d] (%d) stopped by signal %d\n", jid, pid, sig);
    }
}
\end{minted}

\subsection{\texttt{eval}解析函数}

\texttt{eval}函数的功能是解析命令行,分析输入的命令是否合法,判断输入的是否是内置命令,执行指定输入的命令。

\begin{minted}{c}
void eval(char* cmdline) {
    if (cmdline == NULL) {
        return;
    }
    // Local copy of argument list
    char* argv[MAXARGS];
    // Parse the command line and specify if it is a background job
    int bg = parseline(cmdline, argv);
    if (argv[0] == NULL)
        return;
    int state = bg ? BG : FG;
    // Judge if a built-in command is requested
    if (builtin_cmd(argv)) {
        return;
    }

    // Prepare to block SIGCHLD
    sigset_t mask_all, prev_mask;
    sigemptyset(&mask_all);
    sigaddset(&mask_all, SIGCHLD);

    // Fork a child process
    sigprocmask(SIG_BLOCK, &mask_all, &prev_mask);
    pid_t pid = fork();
    if (pid == 0) {
        // Child process
        sigprocmask(SIG_SETMASK, &prev_mask, NULL);
        setpgid(0, 0); // Set the process group ID to the child PID
        if (execve(argv[0], argv, environ) < 0) {
            printf("%s: Command not found\n", argv[0]);
            exit(1);
        }
    } else if (pid > 0) {
        // Parent process

        // Add the job to the job list
        addjob(jobs, pid, state, cmdline);
        sigprocmask(SIG_SETMASK, &prev_mask, NULL);

        if (state == FG) {
            // Wait for the foreground job to terminate
            waitfg(pid);
        } else {
            printf("[%d] (%d) %s", pid2jid(pid), pid, cmdline);
        }
    } else {
        // Fork error
    }
    return;
}
\end{minted}

这段代码首先试图以内置指令的方式执行输入命令;如果不是内置命令,则会执行\texttt{fork}函数创建子进程,子进程会执行\texttt{execve}函数来执行输入的命令,父进程会将子进程添加到作业列表中,并根据输入的命令是否是前台命令来决定是否等待子进程结束。

在\texttt{fork}之前还要注意使用\texttt{sigprocmask}函数来阻塞\texttt{SIGCHLD}信号,以避免在子进程创建时父进程接收到信号导致的竞态错误。

\subsection{执行内置函数}

\texttt{builtin\_cmd}函数的功能是判断输入的命令是否是内置命令,如果是\texttt{quit}指令,直接执行\texttt{exit(0)};如果是\texttt{jobs}指令,调用已经实现好的\texttt{listjobs}函数;如果是\texttt{fg}或\texttt{bg},调用待实现的\texttt{do\_bgfg}函数;如果不是则返回\texttt{0}:

\begin{minted}{c}
int builtin_cmd(char** argv) {
    if (strcmp(argv[0], "quit") == 0) {
        exit(0);
    } else if (strcmp(argv[0], "jobs") == 0) {
        listjobs(jobs);
        return 1;
    } else if (strcmp(argv[0], "bg") == 0 ||
               strcmp(argv[0], "fg") == 0) {
        do_bgfg(argv);
        return 1;
    }
    return 0; /* not a builtin command */
}
\end{minted}

下面还需要实现\texttt{do\_bgfg}函数来处理\texttt{bg}和\texttt{fg}命令:

\begin{minted}{c}
void do_bgfg(char** argv) {
    int jid = 0;
    pid_t pid = 0;
    struct job_t* job;

    // Parse the argument of fg/bg
    if (argv[1] == NULL) {
        printf("%s command requires PID or %%jobid argument\n",
               argv[0]);
        return;
    } else if (argv[1][0] == '%') {
        jid = atoi(&argv[1][1]);
        job = getjobjid(jobs, jid);
        if (job == NULL) {
            printf("%s: No such job\n", argv[1]);
            return;
        }
        pid = job->pid;
    } else if (isdigit(argv[1][0])) {
        pid = atoi(argv[1]);
        job = getjobpid(jobs, pid);
        if (job == NULL) {
            printf("(%d): No such process\n", pid);
            return;
        }
        jid = job->jid;
    } else {
        printf("%s: argument must be a PID or %%jobid\n", argv[0]);
        return;
    }

    // Execute fg/bg
    if (strcmp(argv[0], "bg") == 0) {
        printf("[%d] (%d) %s", jid, pid, job->cmdline);
        job->state = BG;
        kill(-pid, SIGCONT);
    } else if (strcmp(argv[0], "fg") == 0) {
        // printf("[%d] (%d) %s", jid, pid, job->cmdline);
        job->state = FG;
        kill(-pid, SIGCONT);
        waitfg(pid);
    }
    return;
}
\end{minted}

这段代码首先检查了输入指令的合法性,如果不合法会输出相应提示。对于合法的输入,会使用\texttt{kill}函数向命令中的指定进程发送\texttt{SIGCONT}信号来继续执行该进程并且将其状态设置为前台或后台。

\subsection{\texttt{waitfg}函数和信号处理函数}
\label{sec:signal_handlers}

\texttt{waitfg}函数会等待前台进程结束,并对其回收;\texttt{sigchld\_handler}函数会在接收到子进程发送的\texttt{SIGCHLD}信号时进行工作回收,二者逻辑相似,使用刚刚实现的\texttt{job\_reaping}函数即可完成对应逻辑:

\begin{minted}{c}
void waitfg(pid_t pid) {
    int status;
    if (waitpid(pid, &status, WUNTRACED) == pid) {
        job_reaping(pid, status);
    }
    verbose_printf("waitfg: Process (%d) no longer the fg process\n",
                   pid);
}

void sigchld_handler(int sig) {
    verbose_printf("sigchld_handler: entering\n");

    int status;
    pid_t pid;
    while ((pid = waitpid(-1, &status, WNOHANG | WUNTRACED)) > 0) {
        job_reaping(pid, status);
    }

    verbose_printf("sigchld_handler: exiting\n");
}
\end{minted}

读者可能会有疑问,前台进程退出时,两个函数可能都会被调用,是否有可能因为二者都调用了\texttt{job\_reaping}函数而导致错误?答案是不会的,因为在一个进程中只会有一个\texttt{waitpid}函数会获取到子进程的退出状态,无法获取的\texttt{waitpid}函数不会返回进程号,因此只有一个函数会回收前台进程。

实验还要求实现\texttt{SIGINT}和\texttt{SIGTSTP}信号的处理函数,可以使用\texttt{kill}函数向前台进程所在的进程组发送相应信号:

\begin{minted}{c}
void sigint_handler(int sig) {
    verbose_printf("sigint_handler: entering\n");
    pid_t pid = fgpid(jobs);
    if (pid > 0) {
        kill(-pid, SIGINT);
        verbose_printf("sigint_handler: Job (%d) killed\n", pid);
    }
    verbose_printf("sigint_handler: exiting\n");
    return;
}

void sigtstp_handler(int sig) {
    verbose_printf("sigtstp_handler: entering\n");
    pid_t pid = fgpid(jobs);
    if (pid > 0) {
        verbose_printf("sigtstp_handler: Job [%d] (%d) stopped\n",
                       pid2jid(pid), pid);
        kill(-pid, SIGTSTP);
    }
    verbose_printf("sigtstp_handler: exiting\n");
    return;
}
\end{minted}

\subsection{测试}

经测试,实验提供的\texttt{tsh}程序可以正常运行,并且运行结果与实验资源中提供的测试样例一致。

由于实验中实现进程回收的方式与\texttt{tshref}实现的方式可能存在不同,后者的进程回收统一在\texttt{sigchld\_handler}中进行,所以在开启\texttt{-v}选项时,输出的内容会存在差异。由于部分老师要求实现的程序需要与\texttt{tshref}具有一样的行为,在这一要求下,注释掉\texttt{sigchld\_handler}中的\texttt{verbose\_printf}函数,将\texttt{job\_reaping}中输出的日志伪造成\texttt{sigchld\_handler}的输出即可。

\begin{minted}{c}
void job_reaping(pid_t pid, int status) {
    verbose_printf("sigchld_handler: entering\n");
    if (WIFEXITED(status)) {
        deletejob(jobs, pid);
        verbose_printf("sigchld_handler: Job [%d] (%d) deleted\n",
                       pid2jid(pid), pid);
    } else if (WIFSIGNALED(status)) {
        int jid = pid2jid(pid);
        int sig = WTERMSIG(status);
        deletejob(jobs, pid);
        verbose_printf("sigchld_handler: Job [%d] (%d) deleted\n", jid,
                       pid);
        printf("Job [%d] (%d) terminated by signal %d\n", jid, pid,
               sig);
    } else if (WIFSTOPPED(status)) {
        int jid = pid2jid(pid);
        int sig = WSTOPSIG(status);
        struct job_t* job = getjobjid(jobs, jid);
        job->state = ST;
        printf("Job [%d] (%d) stopped by signal %d\n", jid, pid, sig);
    }
    verbose_printf("sigchld_handler: exiting\n");
}
\end{minted}

\section{总结}
\subsection{实验中出现的问题}

\begin{itemize}
    \item 实验过程中Visual Studio Code刚开始一直对实验资源提供的代码提示错误,但又可以正常通过\texttt{make}命令构建。后来发现按照代码规范,要启用GNU提供的扩展功能才能使用\texttt{sigset\_t}等信号相关的类型,在预处理部分添加\texttt{\#define \_GNU\_SOURCE}解决了这一问题。
    \item 在\ref{sec:signal_handlers}节中,实现\texttt{sigchld\_handler}函数时,调用\texttt{waitpid}函数时一开始并没有开启\texttt{WNOHANG}选项,导致\texttt{sigchld\_handler}可能会被阻塞,在查阅\texttt{waitpid}的手册后了解了这一API的基本使用方法,添加了所需选项实现了所需功能。
\end{itemize}

\subsection{心得体会}

通过本次ShellLab实验,我深入理解了Linux系统中进程控制和信号处理的机制。实验要求实现一个简易的Shell,支持前后台作业管理、信号处理以及内置命令的执行,这使我对操作系统的核心概念有了更直观的认识。

在实验过程中,我学会了如何使用\texttt{fork}创建子进程,利用\texttt{execve}执行新程序,以及通过\texttt{waitpid}等待子进程的结束。同时,掌握了如何设置进程组ID,以便正确地向前台作业发送信号。此外,实验中对\texttt{SIGCHLD}、\texttt{SIGINT}和\texttt{SIGTSTP}等信号的处理,使我理解了信号在进程间通信中的重要作用。

通过调试和测试,我体会到编写健壮的系统程序需要细致的思考和严谨的编码习惯。例如,在处理信号时,必须注意阻塞和解除阻塞的时机,以防止竞态条件的发生。实验还让我认识到,良好的代码结构和清晰的逻辑对于实现复杂功能至关重要。

总的来说,ShellLab实验不仅提升了我的编程能力,更加深了我对操作系统原理的理解,为后续学习和实践打下了坚实的基础。

\end{document}